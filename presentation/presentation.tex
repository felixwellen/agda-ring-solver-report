\documentclass[usenames,dvipsnames]{beamer}
\usepackage{pgfpages}
\setbeamertemplate{note page}[plain]
\usepackage{tikz}
\usetikzlibrary{cd}
\usepackage{catchfilebetweentags}
\usepackage{amssymb}
\usepackage{turnstile}
\usepackage{bbm}
\usepackage[greek, english]{babel}
\usepackage{MnSymbol}
\usepackage{stmaryrd}
\usepackage{csquotes}
\newcommand\doubleplus{+\kern-1.3ex+\kern0.8ex}
\newcommand\mdoubleplus{\ensuremath{\mathbin{+\mkern-8mu+}}}
\makeatletter
\newcommand\incircbin
{%
  \mathpalette\@incircbin
}
\newcommand\@incircbin[2]
{%
  \mathbin%
  {%
    \ooalign{\hidewidth$#1#2$\hidewidth\crcr$#1\bigcirc$}%
  }%
}
\newcommand{\oeq}{\ensuremath{\incircbin{=}}}
\makeatother
\usepackage{ucs}
\DeclareUnicodeCharacter{8759}{\ensuremath{\squaredots}}
\DeclareUnicodeCharacter{951}{\textgreek{\texteta}}
\DeclareUnicodeCharacter{737}{\ensuremath{^\text{l}}}
\DeclareUnicodeCharacter{691}{\ensuremath{^\text{r}}}
\DeclareUnicodeCharacter{7523}{\ensuremath{_\text{r}}}
\DeclareUnicodeCharacter{8718}{\ensuremath{\blacksquare}}
\DeclareUnicodeCharacter{957}{\textgreek{\textnu}}
\DeclareUnicodeCharacter{961}{\textgreek{\textrho}}
\DeclareUnicodeCharacter{929}{\textgreek{\textRho}}
\DeclareUnicodeCharacter{954}{\textgreek{\textkappa}}
\DeclareUnicodeCharacter{10214}{\ensuremath{\lsem}}
\DeclareUnicodeCharacter{10215}{\ensuremath{\rsem}}
\DeclareUnicodeCharacter{8857}{\mdoubleplus}
\DeclareUnicodeCharacter{8860}{\oeq}
\DeclareUnicodeCharacter{9043}{\ensuremath{\triangle}}
\DeclareUnicodeCharacter{928}{\textgreek{\textPi}}
\DeclareUnicodeCharacter{922}{\textgreek{\textKappa}}
\DeclareUnicodeCharacter{931}{\textgreek{\textSigma}}
\DeclareUnicodeCharacter{916}{\textgreek{\textDelta}}
\DeclareUnicodeCharacter{8779}{\ensuremath{\backtriplesim}}
\DeclareUnicodeCharacter{8799}{\ensuremath{\stackrel{?}{=}}}
\DeclareUnicodeCharacter{10181}{\ensuremath{\lbag}}
\DeclareUnicodeCharacter{10182}{\ensuremath{\rbag}}
\DeclareUnicodeCharacter{8760}{\ensuremath{-}}
\usepackage[utf8x]{inputenc}
\usepackage[T1]{fontenc}
\usepackage{autofe}
\usepackage[references]{agda}
\usepackage{bbding}
\setlength{\marginparwidth}{2cm}
\usepackage[obeyDraft]{todonotes}
\usepackage{graphicx}
\usepackage{tikz}
\usetikzlibrary{decorations.pathmorphing}
\usetikzlibrary{snakes}
\usetikzlibrary{arrows}
\usepackage{forest}
\usepackage{multicol}
\usetheme{metropolis}
\usepackage{natbib}
\usepackage{bibentry}
\usepackage{amsmath}
\usepackage{tabularx}
\usepackage{adjustbox}

\title{Automatically and Efficiently Illustrating Polynomial Equalities in Agda}
\author{Donnacha Oisín Kidney}
\begin{document}
\bibliographystyle{plain}
\nobibliography{../bibliography.bib}
\maketitle
\begin{frame}
  \frametitle{Agda}
  Agda is both a mathematical formalism and an executable programming language.

  As a programming language, it is similar to Haskell.

  \ExecuteMetaData[Code.tex]{reverse}


  \bibentry{norell_dependently_2008}
\end{frame}
\begin{frame}
  \frametitle{Mathematical Formalisms}
  \begin{block}{Classical}
    Zermelo-Fraenkel Set Theory
  \end{block}

  \begin{block}{Constructivist}
    \medskip

    \begin{columns}
      \column{0.44\textwidth}
      Martin-Löf's Intuitionistic Type Theory
      \column{0.44\textwidth}
      Agda, Idris
    \end{columns}

    \bigskip

    \begin{columns}
      \column{0.44\textwidth}
      Calculus of Constructions
      \column{0.44\textwidth}
      Coq
    \end{columns}
  \end{block}


  \metroset{block=fill}
  \begin{columns}[T,onlytextwidth]
    \column{0.49\textwidth}
    \begin{alertblock}{Axiom of Choice}
      \[\forall P. P \vee \neg P\]
    \end{alertblock}
    \column{0.49\textwidth}
    \begin{alertblock}{Proof by Contradiction}
      \[\forall P. \neg \neg P \rightarrow P\]
    \end{alertblock}
  \end{columns}
\end{frame}
\begin{frame}
  \frametitle{Two Problems}
  \begin{columns}
    \column{0.5\textwidth}
    \begin{block}{Formalisms are too Verbose}
      \bibentry{whitehead_principia_1910}

      \(1+1=2\) proven on page 360.
    \end{block}
    \column{0.5\textwidth}
    \begin{block}{Automation is Untrustworthy}
      \bibentry{appel_solution_1977}
    \end{block}
  \end{columns}
\end{frame}
\begin{frame}[fragile]
  \frametitle{This Project}
  \begin{figure}[t]
    \ExecuteMetaData[../Introduction.tex]{lemma}
    \vspace{-10pt}
  \end{figure}

  \begin{minipage}[b]{0.5\textwidth}
    \begin{figure}[h!]
      \hspace*{-80pt}\resizebox{0.8\textwidth}{!}{%
        \begin{minipage}{\textwidth}
          \ExecuteMetaData[../Introduction.tex]{proof}
        \end{minipage}
      }
      \caption{A Tedious Proof}
    \end{figure}%
  \end{minipage}%
  \begin{minipage}[b]{0.5\textwidth}
  \begin{figure}[h]
    \resizebox{\textwidth}{!}{%
      \begin{minipage}{\textwidth}
        \ExecuteMetaData[../Introduction.tex]{solver}
      \end{minipage}
    }
    \caption{Our Solver}
  \end{figure}
  \end{minipage}
\end{frame}
\begin{frame}
  \frametitle{The Algorithm}
  Horner normal form
\end{frame}
\begin{frame}
  \frametitle{The Interface}
  Reflection
\end{frame}
\begin{frame}
  \frametitle{Didactic Solutions}
  Perfectly suited
\end{frame}

\begin{frame}[standout]
  Questions?
\end{frame}
\end{document}
