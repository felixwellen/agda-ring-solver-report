\documentclass[draft,usenames,dvipsnames]{beamer}
\usepackage{tikz}
\usetikzlibrary{cd}
\usepackage{catchfilebetweentags}
\usepackage{amssymb}
\usepackage{turnstile}
\usepackage{bbm}
\usepackage[greek, english]{babel}
\usepackage{MnSymbol}
\usepackage{stmaryrd}
\usepackage{csquotes}
\newcommand\doubleplus{+\kern-1.3ex+\kern0.8ex}
\newcommand\mdoubleplus{\ensuremath{\mathbin{+\mkern-8mu+}}}
\makeatletter
\newcommand\incircbin
{%
  \mathpalette\@incircbin
}
\newcommand\@incircbin[2]
{%
  \mathbin%
  {%
    \ooalign{\hidewidth$#1#2$\hidewidth\crcr$#1\bigcirc$}%
  }%
}
\newcommand{\oeq}{\ensuremath{\incircbin{=}}}
\makeatother
\usepackage{ucs}
\DeclareUnicodeCharacter{8759}{\ensuremath{\squaredots}}
\DeclareUnicodeCharacter{951}{\textgreek{\texteta}}
\DeclareUnicodeCharacter{737}{\ensuremath{^\text{l}}}
\DeclareUnicodeCharacter{691}{\ensuremath{^\text{r}}}
\DeclareUnicodeCharacter{7523}{\ensuremath{_\text{r}}}
\DeclareUnicodeCharacter{8718}{\ensuremath{\blacksquare}}
\DeclareUnicodeCharacter{957}{\textgreek{\textnu}}
\DeclareUnicodeCharacter{961}{\textgreek{\textrho}}
\DeclareUnicodeCharacter{929}{\textgreek{\textRho}}
\DeclareUnicodeCharacter{954}{\textgreek{\textkappa}}
\DeclareUnicodeCharacter{10214}{\ensuremath{\lsem}}
\DeclareUnicodeCharacter{10215}{\ensuremath{\rsem}}
\DeclareUnicodeCharacter{8857}{\mdoubleplus}
\DeclareUnicodeCharacter{8860}{\oeq}
\DeclareUnicodeCharacter{9043}{\ensuremath{\triangle}}
\DeclareUnicodeCharacter{928}{\textgreek{\textPi}}
\DeclareUnicodeCharacter{922}{\textgreek{\textKappa}}
\DeclareUnicodeCharacter{931}{\textgreek{\textSigma}}
\DeclareUnicodeCharacter{916}{\textgreek{\textDelta}}
\DeclareUnicodeCharacter{8779}{\ensuremath{\backtriplesim}}
\DeclareUnicodeCharacter{8799}{\ensuremath{\stackrel{?}{=}}}
\DeclareUnicodeCharacter{10181}{\ensuremath{\lbag}}
\DeclareUnicodeCharacter{10182}{\ensuremath{\rbag}}
\usepackage[utf8x]{inputenc}
\usepackage[T1]{fontenc}
\usepackage{autofe}
\usepackage[references]{agda}
\usepackage{bbding}
\setlength{\marginparwidth}{2cm}
\usepackage[obeyDraft]{todonotes}
\usepackage{graphicx}
\usepackage{tikz}
\usetikzlibrary{decorations.pathmorphing}
\usetikzlibrary{snakes}
\usetikzlibrary{arrows}
\usepackage{forest}
\usepackage{multicol}
\usetheme{metropolis}
\usepackage{natbib}
\usepackage{bibentry}
\title{Programming Mathematics in Agda}
\author{Donnacha Oisín Kidney}
\begin{document}
\bibliographystyle{plain}
\nobibliography{../horners-rule.bib}
\section{Dependently Typed Programming}
\begin{frame}[fragile]
  \frametitle{The Curry-Howard Correspondence}
  \begin{figure}
    \centering
    \begin{tikzcd}
      Type    \ar[d] \ar[r, Leftrightarrow] & Proposition \ar[d] \\
      Program \ar[r, Leftrightarrow]        & Proof
    \end{tikzcd}
  \end{figure}

  \bibentry{wadler_propositions_2015-1}
\end{frame}
\subsection{For Programmers}
\begin{frame}
  Types are (usually):
  \begin{itemize}
    \item \(\AgdaDatatype{Int}\)
    \item \(\AgdaDatatype{String}\)
    \item ...
  \end{itemize}

  How are these propositions?
\end{frame}
\begin{frame}[fragile]
  \frametitle{Existential Proofs}
  \begin{columns}[T]
    \column{0.5\textwidth}
    \centering
    So when you see:
    \ExecuteMetaData[BasicTypes.tex]{xint}
    \column{0.5\textwidth}
    \centering
    Think:
    \[\exists. \mathbb{N}\]
  \end{columns}

  \begin{block}{NB}
    We'll see a more powerful and precise version of \(\exists\) later.
  \end{block}

  Proof is ``by example''

  \ExecuteMetaData[BasicTypes.tex]{xprf}
\end{frame}
\begin{frame}[fragile]
  \frametitle{Example ``Proof''}
  Let's start working with a function as if it were a proof.

  The example function we'll choose gets the first element of a list and returns
  it (commonly called \(\AgdaFunction{head}\) in functional programming
  languages).

  Here's the type:
  \ExecuteMetaData[BasicTypes.tex]{headty}
\end{frame}
\begin{frame}
  \frametitle{Basic Syntax}
  \(\AgdaFunction{head}\) is what would be called a ``generic'' function in
  languages like Java.

  In other words, the type \(A\) is not specified in the implementation of the
  function: it just ``takes a list of things, and returns one of those things''.

  In Agda, you must supply the type to the function: the curly brackets mean the
  argument is implicit.
\end{frame}
\begin{frame}
  \frametitle{The Proposition is False!}
  What happens if we call \(\AgdaFunction{head}\) on an empty list? In this
  case, \(\AgdaFunction{head}\) isn't defined.

  In other words, the proposition:

  \ExecuteMetaData[BasicTypes.tex]{headty}

  Is \emph{False}.

  We shouldn't be able to prove this using Agda.
\end{frame}
\begin{frame}[allowframebreaks]
  \frametitle{But Let's Try Anyway}
  Agda functions are defined (usually) with \emph{pattern-matching}. 

  \ExecuteMetaData[BasicTypes.tex]{fib}

  For the natural numbers, we use the Peano numbers, which gives us 2 patterns:
  zero, and successor.

  \framebreak

  For lists, we also have two patterns: the empty list, and the head element
  followed by the rest of the list.

  \ExecuteMetaData[BasicTypes.tex]{length}

  \framebreak

  For \(\AgdaFunction{head}\), then, we can just write the following:

  \ExecuteMetaData[BasicTypes.tex]{head1}

  \begin{block}{\alert{No!}}
    Partial functions aren't allowed!
  \end{block}

  \framebreak

  It might seem like we can't write this function, then, but there is one more
  way we could get around it.

  \ExecuteMetaData[BasicTypes.tex]{head2}

  To disallow \emph{this} kind of thing, we must ensure all functions are
  \emph{total}. For now, assume this means ``terminating''.
\end{frame}
\begin{frame}
  So that's all well and good: there's a proposition which we know isn't true,
  and we've tried to prove it, and failed (as well we should).

  Let's go a little further, though: can we \emph{prove} that
  \(\AgdaFunction{head}\) doesn't exist?
\end{frame}
\begin{frame}
  \frametitle{Falsehood}
  Often it's said that you can't prove negatives in dependently typed
  programming: not true!

  In our case, we'll use the principle of explosion.

  \begin{block}{Principle of Explosion}
    \emph{``Ex falso quodlibet''}: from falsehood, anything.
  \end{block}

  In Agda:

  \ExecuteMetaData[BasicTypes.tex]{false}
\end{frame}
\begin{frame}
  So let's supply a proof of that fact!

  \ExecuteMetaData[BasicTypes.tex]{head-not}

  Here's how the proof works: for falsehood, we need to prove the supplied
  proposition, no matter what it is. If \(\AgdaFunction{head}\) exists, this is
  no problem! Just get the head of a list of proofs of the proposition, which
  can be empty.
\end{frame}
\section{A Polynomial Solver}
\section{The \(p\)-Adics}
\end{document}