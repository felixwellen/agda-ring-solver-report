\documentclass[draft, twocolumn]{article}
\usepackage{cite}
\usepackage{url}
\usepackage{catchfilebetweentags}
\usepackage{amssymb}
\usepackage{turnstile}
\usepackage{bbm}
\usepackage[greek, english]{babel}
\usepackage{MnSymbol}
\newcommand\doubleplus{+\kern-1.3ex+\kern0.8ex}
\newcommand\mdoubleplus{\ensuremath{\mathbin{+\mkern-8mu+}}}
\makeatletter
\newcommand\incircbin
{%
  \mathpalette\@incircbin
}
\newcommand\@incircbin[2]
{%
  \mathbin%
  {%
    \ooalign{\hidewidth$#1#2$\hidewidth\crcr$#1\bigcirc$}%
  }%
}
\newcommand{\oeq}{\ensuremath{\incircbin{=}}}
\makeatother
\usepackage{ucs}
\DeclareUnicodeCharacter{8759}{\ensuremath{\squaredots}}
\DeclareUnicodeCharacter{951}{\textgreek{\texteta}}
\DeclareUnicodeCharacter{737}{\ensuremath{^l}}
\DeclareUnicodeCharacter{691}{\ensuremath{^r}}
\DeclareUnicodeCharacter{8718}{\ensuremath{\blacksquare}}
\DeclareUnicodeCharacter{957}{\textgreek{\textnu}}
\DeclareUnicodeCharacter{961}{\textgreek{\textrho}}
\DeclareUnicodeCharacter{954}{\textgreek{\textkappa}}
\DeclareUnicodeCharacter{10214}{\ensuremath{\lsem}}
\DeclareUnicodeCharacter{10215}{\ensuremath{\rsem}}
\DeclareUnicodeCharacter{8857}{\mdoubleplus}
\DeclareUnicodeCharacter{8860}{\oeq}
\usepackage[utf8x]{inputenc}
\usepackage[T1]{fontenc}
\usepackage{autofe}
\usepackage{agda}
\usepackage{bbding}
\setlength{\marginparwidth}{2cm}
\usepackage[obeyDraft]{todonotes}
\author{D Oisín Kidney}
\title{An Efficient and Flexible Evidence-Providing Polynomial Solver for
  Polynomials in Agda}
\begin{document}
\maketitle
\begin{abstract}
  We provide an efficient implementation of a polynomial solver in the
  programming language Agda, and demonstrate its use in a variety of
  applications.
\end{abstract}
\tableofcontents
\section{Introduction}
Dependently typed languages such as Agda\cite{norell_dependently_2008} and
Coq\cite{the_coq_development_team_2018_1219885} allow programmers to write
machine-checked proofs as programs. They provide a degree of reassurance that
handwritten proofs cannot, and allow for exploration of abstract concepts in a
machine-assisted environment.

We will describe an efficient implementation of an automated prover for
equalities in ring and ring-like structures, and show how it can be extended for
use in settings more exotic than simple equality.
\section{Monoids}
Before describing the ring solver, first we will explain the simpler case of a
monoid solver.

A monoid is a set equipped with a binary operation, \(\bullet\), and a
distinguished element \(\epsilon\), which obeys the laws:
\begin{align}
  x \bullet (y \bullet z) &= (x \bullet y) \bullet z \tag{Associativity} \\
  x \bullet \epsilon      &= x \tag{Left Identity} \\
  \epsilon \bullet x      &= x \tag{Right Identity}
\end{align}
\subsection{Equality Proofs}
Monoids can be represented in Agda in a straightforward way, as a record (see
figure~\ref{mon-def}).
\begin{figure}
  \ExecuteMetaData[Code.tex]{mon-def}
  \caption{The definition of Monoid in the Agda Standard
    Library\cite{danielsson_agda_2018}}
  \label{mon-def}
\end{figure}

These come equipped with their own equivalence relation, according to which
proofs for each of the monoid laws are provided. Using this, we can prove
identities like the one in figure~\ref{mon-ident}.
\begin{figure}[h]
  \ExecuteMetaData[Code.tex]{mon-ident}
  \caption{Example Identity}
  \label{mon-ident}
\end{figure}

While it seems like an obvious identity, the proof is somewhat involved
(figure~\ref{mon-proof}).
\begin{figure}[!h]
  \ExecuteMetaData[Code.tex]{mon-proof}
  \caption{Proof of identity in figure~\ref{mon-ident}}
  \label{mon-proof}
\end{figure}

The syntax mimics that of normal, handwritten proofs: the successive ``states''
of the expression are interspersed with equivalence proofs (in the brackets).
Perhaps surprisingly, the syntax is not built-in: it's simply defined in the
standard library.

Despite the powerful syntax, the proof is mechanical, and it's clear that
similar proofs would become tedious with more variables or more complex algebras
(like rings). Luckily, we can automate the procedure.
\subsection{Canonical Forms}
Automation of equality proofs like the one above can be accomplished by first
rewriting both sides of the equation into a canonical form. This form depends on
the particular algebra used in the pair of expressions. For instance, a suitable
canonical form for monoids is lists.
\ExecuteMetaData[Code.tex]{list-def}

This type can be thought of as an AST for the ``language of lists''. Crucially,
it's equivalent to the ``language of monoids'': this is the language of
expressions written using only variables and the monoid operations, like the
expressions in figure~\ref{mon-ident}. The neutral element and binary operator
have their equivalents in lists: \(\epsilon\) is simply the empty list, whereas
\(\bullet\) is list concatenation.
\ExecuteMetaData[Code.tex]{list-monoid}

We can translate between the language of lists and monoid expressions
\footnote{
  For simplicity's sake, instead of curried functions of \(n\)
  arguments, we'll instead deal with functions which take a vector of length
  \(n\), that refer to each variable by position, using Fin, the type of finite
  sets. Of course these two representations are equivalent, but the translation
  is not directly relevant to what we're doing here: we refer the interested
  reader to the Relation.Binary.Reflection module of Agda's standard
  library\cite{danielsson_agda_2018}.
}
with \(\mu\) and \(\eta\).
\ExecuteMetaData[Code.tex]{list-trans}

We have one half of the equality so far: that of the canonical forms. As such,
we have an ``obvious'' proof of the identity in figure~\ref{mon-ident},
expressed in the list language (figure~\ref{list-obvious}).
\begin{figure}[!h]
  \ExecuteMetaData[Code.tex]{list-obvious}
  \caption{The identity in figure~\ref{mon-ident}, expressed in the list
    language}
  \label{list-obvious}
\end{figure}
\subsection{Homomorphism}
Figure~\ref{list-obvious} gives us a proof of the form:

\begin{equation}
  \label{list-list}
  \text{lhs}_{list} = \text{rhs}_{list}
\end{equation}

What we want, though, is the following:

\begin{equation}
  \label{mon-mon}
  \text{lhs}_{mon} = \text{rhs}_{mon}
\end{equation}

Equation~\ref{list-list} can be used to build equation~\ref{mon-mon}, if we
supply two extra proofs:

\begin{equation}
  \text{lhs}_{mon} \overset{a}{=} \text{lhs}_{list} = \text{rhs}_{list}
  \overset{b}{=} \text{rhs}_{mon}
\end{equation}

The proofs labeled \(a\) and \(b\) are the task of this section.

First, we'll define a concrete AST for the monoid language
(figure~\ref{mon-ast}). It has constructors for each of the monoid operations
(\(\oplus\) and \(\text{e}\) are \(\bullet\) and \(\epsilon\), respectively),
and it's indexed by the number of variables it contains, which are constructed
with \(\nu\). Converting back to an opaque function is accomplished in
figure~\ref{eval-ast}.

\begin{figure}
  \ExecuteMetaData[Code.tex]{mon-ast}
  \caption{The AST for the Monoid Language}
  \label{mon-ast}
\end{figure}
\begin{figure}
  \ExecuteMetaData[Code.tex]{eval-ast}
  \caption{Evaluating the Monoid Language AST}
  \label{eval-ast}
\end{figure}

Finally, then, we must prove the equivalence of the monoid and list languages.
This consists of the following proofs:

\begin{align}
  (\eta x) \mu \rho           &= \left\lsem \nu x \right\rsem \rho      \\
  (x \mdoubleplus y) \mu \rho &= \left\lsem x \oplus y \right\rsem \rho \\
  [] \mu \rho                 &= \left\lsem e \right\rsem \rho
\end{align}
The latter two proofs comprise a monoid homomorphism.
\subsection{Usage}
Combining all of the components above, with some plumbing provided by the
Relation.Binary.Reflection module, we can finally automate the solving of the
original identity in figure~\ref{mon-ident}:
\ExecuteMetaData[Code.tex]{ident-auto-proof}
\subsection{Reflection}
One annoyance of the automated solver is that we have to write the expression we
want to solve twice: once in the type signature, and again in the argument
supplied to solve. Agda can infer the type signature:
\ExecuteMetaData[Code.tex]{ident-infer-proof}
But we would prefer to write the expression in the type signature, and have it
infer the argument to solve, as the expression in the type signature is the
desired equality, and the argument to solve is something of an implementation
detail.

\todo{Fill in reflection section} This inference can be accomplished using
Agda's reflection mechanisms.

\section{Horner Normal Form}
\subsection{Sparse}
\section{Multivariate}
\subsection{Sparse}
\subsection{K}
\section{Setoid Applications}
\subsection{Traced}
\subsection{Isomorphisms}
\subsection{Counterexamples}
\section{The Correct-by-Construction Approach}
\section{Reflection}
\bibliographystyle{IEEEtranS}
\bibliography{horners-rule.bib}
\end{document}