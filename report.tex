\documentclass{article}
\usepackage{cite}
\usepackage{url}
\usepackage{catchfilebetweentags}
\usepackage{amssymb}
\usepackage{turnstile}
\usepackage{bbm}
\usepackage[greek, english]{babel}
\usepackage{MnSymbol}
\usepackage{ucs}
\DeclareUnicodeCharacter{8759}{\ensuremath{\squaredots}}
\usepackage[utf8x]{inputenc}
\usepackage{autofe}
\usepackage{agda}
\begin{document}

There's a particular function on lists that I'm a little obsessed with:

\ExecuteMetaData[Code.tex]{times}

It's an implementation of discrete convolution on lists. Previously I discussed
it in relation to search patterns: it corresponds (somewhat) to breadth-first
search (rather than depth-first).

Here though, I want to talk about its more traditional interpretation: the
multiplication of two polynomials. Indeed, if you write out your polynomial
backwards: 

\begin{align}
&   &   & 2x^2 & + & x    & - &   & 4    & \\
& = &   & 2x^2 & + & 1x^1 & + & - & 4x^0 &  \{ \text{With explicit powers of $x$} \} \\
& = & - & 4x^0 & + & 1x^1 & + &   & 2x^2 & \{ \text{Reversed} \}
\end{align}

\cite{rivas_monoids_2015}

% -- Reflection
% -- Counterexamples
% You can write the exponents in a list:

% ```haskell
% [-4,1,2]
% ```

% And you can multiply them as so:

% ```haskell
% >>> [-4,1,2] <.> [-4,1,2]
% [[(-4,-4)],[(-4,1),(1,-4)],[(-4,2),(1,1),(2,-4)],[(1,2),(2,1)],[(2,2)]]
% ```

% Although the output isn't what we're looking for: that's because it's in a
% somewhat sum-of-products form. We can get it back to something readable like so:

% ```haskell
% >>> run = map (sum . map (uncurry (*)))
% >>> run ([-4,1,2] <.> [-4,1,2])
% [16,-8,-15,4,4]
% ```

% Which is exactly what we want:

% $$16 - 8x -15x^2 + 4x^3 + 4x^4$$

% # So What's The Use?

% This operation on lists, along with its companion in the form of addition:

% ```haskell
% (<+>) :: [a] -> [a] -> [[a]]
% [] <+> ys = map pure ys
% xs <+> [] = map pure xs
% (x:xs) <+> (y:ys) = [x,y] : (xs <+> ys)
% ```

% allow us to interpret lists as polynomials. This kind of thing has a number of
% uses, but for now I'm interested in the "freeness" of this representation. I'm
% going to be a bit fast and loose with the definitions here, but bear with me:
% free objects are "minimal" examples of some class (class can indeed be a Haskell
% class here, but it's more general). The classic example is lists:

% ```haskell
% data List a = Nil | Cons a (List a)
% ```

% These are the free monoid[^Haskell-free-monoid]: it's the *simplest* type which
% adds monoid operations to some `a`. In other words, you can combine lists in an
% associative way:

% [^Haskell-free-monoid]: Technically speaking, they're
%     [not](http://comonad.com/reader/2015/free-monoids-in-haskell/) free monoids
%     in the presence of $\bot$.

% ```haskell
% [] ++ ys = ys
% (x:xs) ++ ys = x : (xs ++ ys)
% -- xs ++ (ys ++ zs) == (xs ++ ys) ++ zs
% ```

% You also get an "empty" element:

% ```haskell
% empty = Nil
% -- empty ++ xs == xs
% -- xs ++ empty = xs
% ```

% And you get *nothing else*. One could imagine a class of commutative monoids,
% for instance, which adds commutativity to the requirement of associativity: any
% free monoid should not be a member of this class, as they'd no longer be a
% "minimal" monoid, because they supported an extra law. In short, free objects of
% some class are those which:

% #. Obey the laws of the class.
% #. Do not obey any further laws which can't be derived from the laws of the
%    class.

% We'll need two more things:

% ```haskell
% pure :: forall a.             a -> List a
% fold :: forall m. Monoid m => List m -> m
% ```

% `pure`{.haskell} here is indeed the same pure as the one from
% `Applicative`{.haskell}, but that's just a coincidence, I'm afraid---we won't be
% using it for anything applicative-y today.

% These two functions will allow us to treat lists as a little language for monoid
% expressions. `pure` will give us variables, and `fold`{.haskell} will let us
% execute the AST.

% The correctness of this language is ensured by the fact that `fold` is a
% *homomorphism*. This means it follows the laws:

% ```haskell
% fold (xs ++ ys) == fold xs <> fold ys
% fold [] == mempty
% ```

% In other words, it doesn't make a difference if you use the monoid operations on
% lists and then `fold`, or if you `fold` and then use the monoid operations
% on the result: you'll still get the same answer.

% # No Really, What's The Use?

% Yeah, yeah. Well, free objects in combination with homomorphisms are probably
% most well-known in Haskell in the context of "extensible effects"
% [@kiselyov_extensible_2013-2].

% Here, though, we're going to use the "AST" way of looking at things to prove
% equality of equations written in the class of the free object.

% In Agda, you have two notions of equality. Definitional equality is the more
% superficial of the two: if two expressions *look* equal, they are.

% ```agda
% x + y ≡ x + y
% ```

% Well, it goes a little further than that: it will run functions to normal form,
% without looking under a lambda. That's why, with the following definition of
% `+`:

% ```agda
% _+_ : ℕ → ℕ → ℕ
% zero + y = y
% suc x + y = suc (x + y)
% ```

% The first function here will typecheck, and the second will not:

% ```agda
% +-identityˡ : ∀ x → 0 + x ≡ x
% +-identityˡ _ = refl

% +-identityʳ : ∀ x → x + 0 ≡ x
% +-identityʳ _ = refl
% ```

% We *can* get the second to typecheck, though, like so:

% ```agda
% +-identityʳ : ∀ x → x + 0 ≡ x
% +-identityʳ zero = refl
% +-identityʳ (suc x) = cong suc (+-identityʳ x)
% ```

% We just need to hold Agda's hand, showing it that all cases do indeed result in
% the correct answer.

% This gets tedious, fast, especially when there are lots of variables involved.
% Consider:

% ```agda
% prop : ∀ w x y z →  w ∙ (((x ∙ ε) ∙ y) ∙ z) ≈ (w ∙ x) ∙ (y ∙ z)
% ```

% The proof is ugly and mechanical:

% ```agda
% prop : ∀ w x y z →  w ∙ (((x ∙ ε) ∙ y) ∙ z) ≈ (w ∙ x) ∙ (y ∙ z)
% prop w x y z =
%   (refl ⟨ ∙-cong ⟩ assoc (x ∙ ε) y z)
%     ⟨ trans ⟩
%   (sym (assoc w (x ∙ ε) (y ∙ z)))
%     ⟨ trans ⟩
%   (refl ⟨ ∙-cong ⟩ identityʳ x ⟨ ∙-cong ⟩ refl)
% ```

% We know it's true, because `∙` is associative, so parentheses don't matter, and
% `ε` is the empty value, so it doesn't affect anything, but it's difficult to
% convince the compiler of that fact. Like the examples with `+` above, the `∙`
% may rely on the values of its arguments to run, so Agda can't go any further
% with the expression.

% Here's where lists come in: they don't rely on their contents *at all* for the
% monoid operations (otherwise they wouldn't be free monoids), so they can run all
% they like. In fact:


% ```agda
% prop : ∀ {a} {A : Set a} {w x y z : A}
%      →  [ w ] ++ ((([ x ] ++ []) ++ [ y ]) ++ [ z ])
%      ≡ ([ w ] ++ [ x ]) ++ ([ y ] ++ [ z ])
% prop = refl
% ```

% Effectively, lists are a normal form for expressions over monoids: if the two
% normalized expressions are equal, then the expressions themselves must be equal.
% In combination with the homomorphism, this can be used to make an automatic
% solver. The implementation is a little weird, but once done you can write stuff
% like this:

% ```agda
% prop : ∀ w x y z → w ∙ (((x ∙ ε) ∙ y) ∙ z) ≈ (w ∙ x) ∙ (y ∙ z)
% prop = solve 4
%   (λ w x y z →  w ⊕ (((x ⊕ id) ⊕ y) ⊕ z) ⊜ (w ⊕ x) ⊕ (y ⊕ z))
%   refl
% ```

% As it turns out, where lists exhibit this normalizing behaviour for monoids,
% polynomials (as described above) exhibit the same behaviour for *rings*.

% # Implementing a Ring Solver

% Agda [has a ring
% solver](https://agda.github.io/agda-stdlib/Algebra.Solver.Ring.html) in the
% standard library already, which we'll be using a little to help our
% implementation. Mainly, though, we'll use "Proving Equalities in a Commutative
% Ring Done Right in Coq" [@hutchison_proving_2005]. First things first, then, we
% need to nail down the representation we're going to use.

% # Horner Normal Form

% Our first stab at a representation will start off like this:

% ```agda
% open import Algebra

% module Polynomials {a} (coeff : RawRing a) where

% open RawRing coeff
% ```

% The entire module is parameterized over whatever ring we'll end up using:
% `Carrier` is the type of the ring itself. In Haskell, this kind of thing might
% use a type class: here, we pass in a record which contains the type, functions,
% and laws we're interested in. This makes it easier to swap in different rings
% for the same type, and it's the standard Agda way of doing things. The `RawRing`
% type is just a record, defined like this:

% ```agda
% record RawRing c : Set (suc c) where
%   infix  8 -_
%   infixl 7 _*_
%   infixl 6 _+_
%   field
%     Carrier : Set c
%     _+_     : Op₂ Carrier
%     _*_     : Op₂ Carrier
%     -_      : Op₁ Carrier
%     0#      : Carrier
%     1#      : Carrier
% ```

% `Raw` refers to the fact that it doesn't carry any proofs. `Op₂ A` is a type
% synonym for `A → A → A`.

% Now, for the actual functions:

% ```agda
% open import Data.List as List using (List; _∷_; [])

% Poly : Set a
% Poly = List Carrier

% _⊞_ : Poly → Poly → Poly
% [] ⊞ ys = ys
% (x ∷ xs) ⊞ [] = x ∷ xs
% (x ∷ xs) ⊞ (y ∷ ys) = x + y ∷ xs ⊞ ys

% _⊠_ : Poly → Poly → Poly
% [] ⊠ ys = []
% (x ∷ xs) ⊠ [] = []
% (x ∷ xs) ⊠ (y ∷ ys) =
%   x * y ∷ (List.map (x *_) ys ⊞ (xs ⊠ (y ∷ ys)))
% ```


% # Gaps

% As it stands, the above representation has two problems:

% Redundancy

% :   The representation suffers from the problem of trailing zeroes. In other
%     words, the polynomial $2x$ could be represented by any of the following:
  
%     ```agda
%     [0, 2]
%     [0, 2, 0]
%     [0, 2, 0, 0]
%     [0, 2, 0, 0, 0, 0, 0]
%     ```
    
%     This is a problem for a solver: the whole *point* is that equivalent
%     expressions are represented the same way.

% Inefficiency

% :   Expressions will tend to have large gaps, full only of zeroes. Something
%     like $x^5$ will be represented as a list with 6 elements, only the last one
%     being of interest. Since addition is linear in the length of the list, and
%     multiplication quadratic, this is a major concern.

% In the Coq implementation [@hutchison_proving_2005], the problem is addressed
% primarily from the efficiency perspective: they add a field for the "power
% index". For our case, we'll just store a list of pairs, where the second element
% of the pair is the power index.

% A brief aside: in the paper, the following power index:

% ```agda
% (c , i) ∷ P
% ```

% Is said to be the representation of:

% $$ P \times X^i + c $$

% This didn't make a huge amount of sense to me, though. Since we're representing
% the preceding gap, I decided to go with the following translation:

% $$ (P \times X + c) * X^i $$

% By way of example, the polynomial:

% $$ 3 + 2x² + 4x⁵ + 2x⁷ $$

% Will be represented as:

% ```agda
% [(3,0),(2,1),(4,2),(2,1)]
% ```

% Or, mathematically:

% $$ x^0 (3 + x x^1 (2 + x x^2 * (4 + x x^1 (2 + x 0)))) $$


% Also, we can kill two birds with one stone here: if we disallow zeroes
% *entirely* in the list, we lose the gaps, and we also get a unique
% representation. Finally, we have a representation:

% ```agda
% infixr 5 0≠_
% record Coeff : Set (a ⊔ ℓ) where
%   inductive
%   constructor 0≠_
%   field
%     coeff : Carrier
%     .{coeff≠0} : ¬ Zero-C coeff

% Poly : Set (a ⊔ ℓ)
% Poly = List (Coeff × ℕ)
% ```

% The coefficients are ensured to be nonzero, as promised. We allow the user to
% supply an "Is zero" property, which is what we use to ensure every coefficient
% is indeed not zero. I've also marked the proof as irrelevant (that's the dot
% before it), so that it's not computed at runtime.

% To make things easier, we can define a normalizing variant of `∷`:

% ```agda
% infixr 8 _⍓_
% _⍓_ : Poly → ℕ → Poly
% [] ⍓ i = []
% ((x , j) ∷ xs) ⍓ i = (x , j ℕ.+ i) ∷ xs

% infixr 5 _∷↓_
% _∷↓_ : (Carrier × ℕ) → Poly → Poly
% (x , i) ∷↓ xs with zero-c? x
% ... | yes p = xs ⍓ suc i
% ... | no ¬p = (0≠_ x {¬p} , i) ∷ xs
% ```

% # Addition

% Our addition and multiplication functions will need to properly deal with the
% new gapless formulation. First things first, we'll need a way to match the power
% indices. We can use a function from @mcbride_view_2004 to do so:

% ```agda
% data Ordering : ℕ → ℕ → Set where
%   less    : ∀ m k → Ordering m (suc (m + k))
%   equal   : ∀ m   → Ordering m m
%   greater : ∀ m k → Ordering (suc (m + k)) m

% compare : ∀ m n → Ordering m n
% compare zero    zero    = equal   zero
% compare (suc m) zero    = greater zero m
% compare zero    (suc n) = less    zero n
% compare (suc m) (suc n) with compare m n
% compare (suc .m)           (suc .(suc m + k)) | less    m k = less    (suc m) k
% compare (suc .m)           (suc .m)           | equal   m   = equal   (suc m)
% compare (suc .(suc m + k)) (suc .m)           | greater m k = greater (suc m) kv
% ```

% This is a classic example of a "leftist" function: after pattern matching on
% one of the constructors of `Ordering`, it gives you information on type
% variables to the *left* of the pattern. In other words, when you run the
% function on some variables, the result of the function will give you
% information on its arguments.

% # Suspicion

% While the gapless representation is *meant* to be more efficient, that compare
% function is linear in the size of its smaller argument, so I don't see
% how---with peano numbers, at any rate---we have actually improved complexity.
% We've definitely reduced the number of operations on the underlying semiring,
% and there may be efficiency gains elsewhere (multiplication seems a good bit
% faster), but I wonder if there's a way to use a more efficient version of
% `compare`?

% There are
% [seven](https://agda.readthedocs.io/en/v2.5.4/language/built-ins.html#functions-on-natural-numbers)
% functions on natural numbers which Agda will replace with their equivalents on
% Haskell's `Integer`. The three of interest here are:

% ```agda
% _-_ : N → ℕ → ℕ
% n     - zero  = n
% zero  - suc m = zero
% suc n - suc m = n - m
% {-# BUILTIN NATMINUS _-_ #-}

% _==_ : ℕ → ℕ → Bool
% zero  == zero  = true
% suc n == suc m = n == m
% _     == _     = false
% {-# BUILTIN NATEQUALS _==_ #-}

% _<_ : ℕ → ℕ → Bool
% _     < zero  = false
% zero  < suc _ = true
% suc n < suc m = n < m
% {-# BUILTIN NATLESS _<_ #-}
% ```

% The implementation you see is kind of a lie: at runtime, those functions are
% replaced by their faster, unverified implementations. This is good news for us,
% though: we can try reimplement `compare` using them, and we should get a much
% quicker function. Our first try ends in failure:

% ```agda
% compare : (n m : ℕ) → Ordering n m
% compare n m with n < m
% compare n m | true = less n (m - n - 1)
% compare n m | false with n == m
% compare n m | false | false = greater m (n - m - 1)
% compare n m | false | true = equal m
% ```

% Agda complains:

% > `suc (n + (m - n - 1)) != m of type ℕ`

% Hmm. Evidently, we need to prove something. Here's what that proof might look
% like:

% ```agda
% less-hom : ∀ n m → ((n < m) ≡ true) → m ≡ suc (n + (m - n - 1))
% less-hom zero zero ()
% less-hom zero (suc m) _ = refl
% less-hom (suc n) zero ()
% less-hom (suc n) (suc m) n<m = cong suc (less-hom n m n<m)
% ```

% Wait---have we just given up efficiency here? Well, no, because we can
% [*erase*](https://agda.github.io/agda-stdlib/Relation.Binary.PropositionalEquality.TrustMe.html)
% the proof, which hopefully mean it isn't computed at runtime:

% ```agda
% eq-hom : ∀ n m → ((n == m) ≡ true) → n ≡ m
% eq-hom zero zero _ = refl
% eq-hom zero (suc m) ()
% eq-hom (suc n) zero ()
% eq-hom (suc n) (suc m) n≡m = cong suc (eq-hom n m n≡m)

% gt-hom : ∀ n m → ((n < m) ≡ false) → ((n == m) ≡ false) → n ≡ suc (m + (n - m - 1))
% gt-hom zero zero n<m ()
% gt-hom zero (suc m) () n≡m
% gt-hom (suc n) zero n<m n≡m = refl
% gt-hom (suc n) (suc m) n<m n≡m = cong suc (gt-hom n m n<m n≡m)

% compare : (n m : ℕ) → Ordering n m
% compare n m with n < m  | inspect (_<_ n) m
% ... | true  | [ n<m ] rewrite erase (less-hom n m n<m) = less n (m - n - 1)
% ... | false | [ n≮m ] with n == m | inspect (_==_ n) m
% ... | true  | [ n≡m ] rewrite erase (eq-hom n m n≡m) = equal m
% ... | false | [ n≢m ] rewrite erase (gt-hom n m n≮m n≢m) = greater m (n - m - 1)
% ```

% # Termination & Redundancy

% Totality and nontermination isn't actually a problem I've encountered much in
% Agda, but it does occasionally come up. Even when it does, the complaints often
% (not always, but often) point towards redundancy, rather than necessary
% complexity. For instance, the straightforward translation of `⊞` doesn't pass:

% ```agda
% _⊞_ : Poly → Poly → Poly
% _⊞_ [] ys = ys
% _⊞_ (x ∷ xs) [] = x ∷ xs
% _⊞_ ((x , i) ∷ xs) ((y , j) ∷ ys) with compare i j
% ... | less    .i k = (x , i) ∷ xs ⊞ ((y , k) ∷ ys)
% ... | equal   .i   = (coeff x + coeff y , i) ∷↓ (xs ⊞ ys)
% ... | greater .j k = (y , j) ∷ ((x , k) ∷ xs) ⊞ ys
% ```

% Why? Well because the arguments passed in the recursive calls aren't strictly
% smaller---or, at least, Agda can't see that they are. You see, *we* know that
% (for instance) the `k` passed in is smaller than the `j` before it, but it's not
% trivial to show that, so Agda complains.

% One trick to make termination more obvious is to eliminate any redundancy in
% the code. For instance, the first clause above checks if the left-hand-side list
% is empty: but when we call back to the function in the `greater` clause, we
% should be able to skip that check. Here's the strategy: split every clause which
% checks for some condition into its own function, and then call the correct
% function when you know a condition must be passed. 

% ```agda
% mutual
%   infixl 6 _⊞_
%   _⊞_ : Poly → Poly → Poly
%   [] ⊞ ys = ys
%   ((x , i) ∷ xs) ⊞ ys = ⊞-zip-r x i xs ys

%   ⊞-zip-r : Coeff → ℕ → Poly → Poly → Poly
%   ⊞-zip-r x i xs [] = (x , i) ∷ xs
%   ⊞-zip-r x i xs ((y , j) ∷ ys) = ⊞-zip (compare i j) x xs y ys

%   ⊞-zip : ∀ {p q}
%         → Ordering p q
%         → Coeff
%         → Poly
%         → Coeff
%         → Poly
%         → Poly
%   ⊞-zip (less    i k) x xs y ys = (x , i) ∷ ⊞-zip-r y k ys xs
%   ⊞-zip (greater j k) x xs y ys = (y , j) ∷ ⊞-zip-r x k xs ys
%   ⊞-zip (equal   i  ) (0≠ x) xs (0≠ y) ys =
%     (x + y , i) ∷↓ (xs ⊞ ys)
% ```

% And it works! This function is structurally terminating. You'll notice that
% effectively every sum type gets its own function: the first only checks if its
% left argument is cons or nil, the second its right, and the third checks the
% result of the comparison. This should also make the function more efficient:
% we're effectively manually performing call-pattern specialization
% [@jones_call-pattern_2007]. I wonder if that optimization can be used in the
% termination checker?

% # Multiplication

% The version of multiplication I began this article with was an inlined version
% of shift and add. What I'm going to write here is the same, but not inlined:
% this will allow us to use `⊞` in the definition, and consequently, in the
% proofs:

% ```agda
% infixl 7 _⋊_
% _⋊_ : Carrier → Poly → Poly
% _⋊_ x = foldr ⋊-step []
%   where
%   ⋊-step : Coeff × ℕ → Poly → Poly
%   ⋊-step (0≠ y , i) ys = (x * y , i) ∷↓ ys

% infixl 7 _⊠_
% _⊠_ : Poly → Poly → Poly
% xs ⊠ [] = []
% xs ⊠ ((0≠ y , j) ∷ ys) = foldr ⊠-step [] xs ⍓ j
%   where
%   ⊠-step : Coeff × ℕ → Poly → Poly
%   ⊠-step (0≠ x , i) xs = (x * y , i) ∷↓ (x ⋊ ys ⊞ xs ⊠ ys)
% ```

% # Binary

% Before we continue with the polynomials, we can take a brief detour to talk
% about binary numbers in proof assistants. Quite often a list of booleans will be
% used (instead of the normal Peano encoding) to improve the efficiency of
% manipulation, while maintaining some of the ability to write proofs. Also, some
% [data structures mimic the structure of binary
% numbers](2017-04-23-verifying-data-structures-in-haskell-lhs.html), meaning that
% the proofs can often carry over.

% One of the problems that shows up in implementations is the same as our problem
% above: redundant zeroes. There are a
% [number](https://lists.chalmers.se/pipermail/agda/2018/010379.html) of
% [ways](http://www.botik.ru/pub/local/Mechveliani/binNat/) to get around this,
% but here we see a pleasingly simple one: represent it as a sparse polynomial!
% There's no need for pairs, since there's only one possible coefficient (1, since
% we've disallowed zeroes). So all we have instead is a list of the number of
% zeroes between each 1:

% ```agda
% 0  = []
% 52 = 001011 = [2,1,0]
% 4  = 001    = [2]
% 5  = 101    = [0,1]
% 10 = 0101   = [1,1]

% Bin : Set
% Bin = List ℕ
% ```

% These are unique representations, and the functions on them are quite simple:

% ```agda
% incr′ : ℕ → Bin → Bin
% incr″ : ℕ → ℕ → Bin → Bin

% incr′ i [] = i ∷ []
% incr′ i (x ∷ xs) = incr″ i x xs

% incr″ i zero xs = incr′ (suc i) xs
% incr″ i (suc x) xs = i ∷ x ∷ xs

% incr : Bin → Bin
% incr = incr′ 0

% infixl 6 _+_
% _+_ : Bin → Bin → Bin
% [] + ys = ys
% (x ∷ xs) + ys = +-zip-r x xs ys
%   where
%   +-zip   :     ∀ {x y} → Ordering x y → Bin → Bin → Bin
%   ∔-zip   : ℕ → ∀ {i j} → Ordering i j → Bin → Bin → Bin
%   +-zip-r :     ℕ → Bin → Bin → Bin
%   ∔-zip-r : ℕ → ℕ → Bin → Bin → Bin
%   ∔-cin   : ℕ → Bin → Bin → Bin
%   ∔-zip-c : ℕ → ℕ → ℕ → Bin → Bin → Bin

%   +-zip (less    i k) xs ys = i ∷ +-zip-r k ys xs
%   +-zip (equal   k  ) xs ys = ∔-cin (suc k) xs ys
%   +-zip (greater j k) xs ys = j ∷ +-zip-r k xs ys

%   +-zip-r x xs [] = x ∷ xs
%   +-zip-r x xs (y ∷ ys) = +-zip (compare x y) xs ys

%   ∔-cin i [] = incr′ i
%   ∔-cin i (x ∷ xs) = ∔-zip-r i x xs

%   ∔-zip-r i x xs [] = incr″ i x xs
%   ∔-zip-r i x xs (y ∷ ys) = ∔-zip i (compare y x) ys xs

%   ∔-zip-c c zero k xs ys = ∔-zip-r (suc c) k xs ys
%   ∔-zip-c c (suc i) k xs ys = c ∷ i ∷ +-zip-r k xs ys

%   ∔-zip c (less    i k) xs ys = ∔-zip-c c i k ys xs
%   ∔-zip c (greater j k) xs ys = ∔-zip-c c j k xs ys
%   ∔-zip c (equal     k) xs ys = c ∷ ∔-cin k xs ys

% pow : ℕ → Bin → Bin
% pow i [] = []
% pow i (x ∷ xs) = (x ℕ.+ i) ∷ xs

% infixl 7 _*_
% _*_ : Bin → Bin → Bin
% _*_ [] _ = []
% _*_ (x ∷ xs) = pow x ∘ foldr (λ y ys → y ∷ xs + ys) []
% ```

% # Multivariate Polynomials

% Up until now our polynomial has been an expression in just one variable. For it
% to be truly useful, though, we'd like to be able to extend it to many: luckily
% there's a well-known isomorphism we can use to extend our earlier
% implementation. A multivariate polynomial is one where its coefficients are
% polynomials with one fewer variable [@cheng_functional_2018]. In Agda, this
% looks like the following:

% ```agda
% mutual
%   Poly : ℕ → Set ℓ
%   Poly zero = Lift ℓ Carrier
%   Poly (suc n) = Coeffs n

%   Coeffs : ℕ → Set ℓ
%   Coeffs n = List (Coeff n × ℕ)

%   infixr 5 0≠_
%   record Coeff (n : ℕ) : Set ℓ where
%     inductive
%     constructor 0≠_
%     field
%       poly : Poly n
%       .{poly≠0} : ¬ Zero n poly

%   Zero : ∀ n → Poly n → Set ℓ
%   Zero zero (lift x) = Zero-C x
%   Zero (suc n) [] = Lift ℓ ⊤
%   Zero (suc n) (x ∷ xs) = Lift ℓ ⊥
% ```

% The addition function looks similar to before, with an extra first helper to
% check if the polynomial is constant:

% ```agda
% mutual
%   infixl 6 _⊞_
%   _⊞_ : ∀ {n} → Poly n → Poly n → Poly n
%   _⊞_ {zero} (lift x) (lift y) = lift (x + y)
%   _⊞_ {suc n} = ⊞-coeffs

%   ⊞-coeffs : ∀ {n} → Coeffs n → Coeffs n → Coeffs n
%   ⊞-coeffs [] ys = ys
%   ⊞-coeffs ((x , i) ∷ xs) ys = ⊞-zip-r x i xs ys

%   ⊞-zip-r : ∀ {n} → Coeff n → ℕ → Coeffs n → Coeffs n → Coeffs n
%   ⊞-zip-r x i xs [] = (x , i) ∷ xs
%   ⊞-zip-r x i xs ((y , j) ∷ ys) = ⊞-zip (compare i j) x xs y ys

%   ⊞-zip : ∀ {p q n}
%         → Ordering p q
%         → (x : Coeff n)
%         → Coeffs n
%         → (y : Coeff n)
%         → Coeffs n
%         → Coeffs n
%   ⊞-zip (less    i k) x xs y ys = (x , i) ∷ ⊞-zip-r y k ys xs
%   ⊞-zip (greater j k) x xs y ys = (y , j) ∷ ⊞-zip-r x k xs ys
%   ⊞-zip (equal   i  ) (0≠ x) xs (0≠ y) ys =
%     (x ⊞ y , i) ∷↓ (⊞-coeffs xs ys)
% ```

% However, there is opportunity for optimization here, as again is exploited in
% the Coq implementation. We actually have a similar kind of gap for nesting as we
% do for exponentiation: consider a polynomial with 4 variables, $W, X, Y, Z$. The
% representation will be something like this:

% ```agda
% Poly of [W,X,Y,Z] with coefficients C = 
%   Poly of W with coefficients (
%     Poly of X with coefficients (
%       Poly of Y with coefficients (
%         Poly of Z with coefficients C)))
% ```

% Therefore, if we want to represent the expression $Z^2$, we're going to need
% three levels of nesting before we get to what we need.

% # Gapless Nesting

% We'll start by attempting a similar approach to removing the exponent gaps.
% Straight away, though, we run into a problem: the poly type is *indexed* by the
% number of variables it contains. So the overall type will have to correspond to
% the gap size in some way. The direct way to encode that would be something like
% this:

% ```agda
% data Poly : ℕ → Set (a ⊔ ℓ) where
%   _Π_ : (gap : ℕ) → ∀ {i} → FlatPoly i → Poly (suc (gap ℕ.+ i))
% ```

% Where `FlatPoly` is effectively the gappy type we had earlier. If you actually
% tried to use this type, though, you'd run into issues:

% ```agda
% _⊞_ : ∀ {n} → Poly n → Poly n → Poly n
% (gap Π x) ⊞ ys = {!!}
% ```

% Try to pattern match on `ys` and you'll get the following error:

% > I'm not sure if there should be a case for the constructor `_Π_`,
% > because I get stuck when trying to solve the following unification
% > problems (inferred index `≟` expected index):
% >
% > ```agda
% >   suc (gap₂ ℕ.+ i₁) ≟ suc (gap₁ ℕ.+ i)
% > ```
% >
% > when checking that the expression ? has type
% >
% > ```agda
% >   Poly (suc (gap ℕ.+ ;i))
% > ```

% Why aren't you sure, Agda?! There is a case for it! I know there is!

% The problem is that Agda is trying to unify something with a *function*, rather
% than constructors. Avoiding this problem is known as "Don't touch the green
% slime!" [@mcbride_polynomial_2018]:

% > When combining prescriptive and descriptive indices, ensure both are in
% > constructor form. Exclude defined functions which yield difficult
% > unification problems.

% So we're going to have to do something else.

% # Storing Inequalities

% Since all we need to know about the nested polynomial is that it does indeed
% have a smaller number of variables than the outer, we can store that fact in a
% proof:

% ```agda
% infixl 6 _Π_
% record Poly (n : ℕ) : Set (a ⊔ ℓ) where
%   inductive
%   constructor _Π_
%   field
%     {i} : ℕ
%     flat  : FlatPoly i
%     i≤n   : i ≤ n
% ```

% We won't get any of the problems above using this approach, and we can go ahead
% with the rest of the implementation. Like with the exponents, we should store a
% proof that this nesting structure is as compact as possible: that it's in normal
% form. There are two cases when a polynomial isn't in normal form: when it's an
% empty list (it could instead be the constant 0), or when it only has one
% constant coefficient. We can express this, like with `Zero`, in a function:

% ```agda
% mutual
%   infixl 6 _Π_
%   record Poly (n : ℕ) : Set (a ⊔ ℓ) where
%     inductive
%     constructor _Π_
%     field
%       {i} : ℕ
%       flat  : FlatPoly i
%       i≤n   : i ≤ n

%   data FlatPoly : ℕ → Set (a ⊔ ℓ) where
%     Κ : Carrier → FlatPoly 0
%     Σ : ∀ {n} → (xs : Coeffs n) → .{xn : Norm xs} → FlatPoly (suc n)

%   infixl 6 _Δ_
%   record CoeffExp (i : ℕ) : Set (a ⊔ ℓ) where
%     inductive
%     constructor _Δ_
%     field
%       coeff : Coeff i
%       pow   : ℕ

%   Coeffs : ℕ → Set (a ⊔ ℓ)
%   Coeffs n = List (CoeffExp n)

%   infixl 6 _≠0
%   record Coeff (i : ℕ) : Set (a ⊔ ℓ) where
%     inductive
%     constructor _≠0
%     field
%       poly : Poly i
%       .{poly≠0} : ¬ Zero poly

%   Zero : ∀ {n} → Poly n → Set ℓ
%   Zero (Κ x       Π _) = Zero-C x
%   Zero (Σ []      Π _) = Lift ℓ ⊤
%   Zero (Σ (_ ∷ _) Π _) = Lift ℓ ⊥

%   Norm : ∀ {i} → Coeffs i → Set
%   Norm []                  = ⊥
%   Norm (_ Δ zero  ∷ [])    = ⊥
%   Norm (_ Δ zero  ∷ _ ∷ _) = ⊤
%   Norm (_ Δ suc _ ∷ _)     = ⊤
% ```

% # Choosing the Inequality

% There are multiple ways to express `x ≤ y` in Agda: in the standard library,
% three of them are defined for natural numbers.

% ## Option 1: The Standard Way

% The first definition of `≤` is as follows:

% ```agda
% data _≤_ : ℕ → ℕ → Set where
%    z≤n : ∀ {n}                 → zero  ≤ n
%    s≤s : ∀ {m n} (m≤n : m ≤ n) → suc m ≤ suc n
% ```

% It actually worked fine, for a bit, until I realized that I had actually made
% the time complexity *worse* by using this encoding of gaps. To understand why,
% remember the addition function above with the gapless exponent encoding. For it to
% work, we needed to compare the gaps, and proceed based on that. We'll need to do
% a similar comparison on variable counts for this gapless encoding. However, we
% don't store the *gaps* now, we store the number of variables in the nested
% polynomial. Consider the following sequence of nestings:

% ```agda
% (5 ≤ 6), (4 ≤ 5), (3 ≤ 4), (1 ≤ 3), (0 ≤ 1)
% ```

% The outer polynomial has 6 variables, but it has a gap to its inner polynomial
% of 5, and so on. The comparisons will be made on 5, 4, 3, 1, and 0. Like
% repeatedly taking the length of the tail of a list, this is quadratic. There
% must be a better way.

% ## Option 2: With Refl

% Once you realize we need to be comparing the gaps and not the tails, the third
% encoding of `≤` jumps out:

% ```agda
% record _≤″_ (m n : ℕ) : Set where
%   constructor less-than-or-equal
%   field
%     {k}   : ℕ
%     proof : m + k ≡ n
% ```

% It stores the gap *right there*: in `k`!

% Unfortunately, though, we're still stuck. While you can indeed run your
% comparison on `k`, you're not left with much information about the rest. Say,
% for instance, you find out that two respective `k`s are equal. What about the
% `m`s? Of course, you *can* show that they must be equal as well, but it requires
% a proof. Similarly in the less-than or greater-than cases: each time, you need
% to show that the information about `k` corresponds to information about `m`.
% Again, all of this can be done, but it all requires propositional proofs, which
% are messy, and slow. Erasure is an option, but I'm not sure of the correctness
% of that approach.

% ## Option 3

% What we really want is to *run* the comparison function on the gap, but get the
% result on the tail. Turns out we can do exactly that with the following:

% ```agda
% infix 4 _≤_
% data _≤_ (m : ℕ) : ℕ → Set where
%   m≤m : m ≤ m
%   ≤-s : ∀ {n} → (m≤n : m ≤ n) → m ≤ suc n
% ```

% (This is a rewritten version of `_≤′_` from Data.Nat.Base).

% While this structure stores the same information as `_≤_`, it does so by
% induction on the *gap*. That structure can be used to write a comparison
% function which was linear in the size of the gap (even though it was comparing
% the length of the tail):

% ```agda
% data Ordering : ℕ → ℕ → Set where
%   less    : ∀ {n m} → n ≤ m → Ordering n (suc m)
%   greater : ∀ {n m} → m ≤ n → Ordering (suc n) m
%   equal   : ∀ {n}           → Ordering n n

% ≤-compare : ∀ {i j n}
%           → (i≤n : i ≤ n)
%           → (j≤n : j ≤ n)
%           → Ordering i j
% ≤-compare m≤m m≤m = equal
% ≤-compare m≤m (≤-s m≤n) = greater m≤n
% ≤-compare (≤-s m≤n) m≤m = less m≤n
% ≤-compare (≤-s i≤n) (≤-s j≤n) = ≤-compare i≤n j≤n
% ```

% A few things to note here:

% #. The `≤-compare` function is one of those reassuring ones for which Agda can
% automatically fill in the implementation from the type.
% #. This function looks somewhat similar to the one for comparing `ℕ` in Data.Nat,
% and as a result, the "matching" logic for degree and number of variables began
% to look similar.

% # Irrelevance and K

% I proceeded happily with the new proof, and the functions began to materialize
% mechanically. The proofs did, too, until I was asked to prove the following:

% ```agda
% ∀ {i n}
% → (x : i ≤ n)
% → (y : i ≤ n)
% → ∀ xs Ρ
% → Σ⟦ xs ⟧ (drop-1 x Ρ) ≈ Σ⟦ xs ⟧ (drop-1 y Ρ)
% ```

% I've already proven that both sides have the same polynomials, and the same
% variables. All I needed to do now was show that the `drop-1` functions behaved
% the same. These simply drop the gap from the front of the supplied vector, so
% the nested polynomial gets the variables it needs.

% It seems like it shouldn't be a problem: after all, every inequality proof is
% *irrelevant*: its structure is entirely determined by its type. By all
% accounts, we should be able to prove something like this:

% ```agda
% irrel : ∀ {i  n}
%       → (x : i ≤ n)
%       → (y : i ≤ n)
%       → x ≡ y
% ```

% If you go ahead and try it, though, with `--without-K` turned on, you'll get the
% following error when you try pattern match on `y` in the first line:

% ```agda
% irrel m≤m y = {!!}
% irrel (≤-s x) y = {!!}
% ```

% > I'm not sure if there should be a case for the constructor `m≤m`,
% > because I get stuck when trying to solve the following unification
% > problems (inferred index `≟` expected index):
% > ```agda
% >   i ≟ i
% > ```
% > Possible reason why unification failed:
% >
% > > Cannot eliminate reflexive equation `i = i` of type `ℕ` because K has
% >   been disabled.
% >
% > when checking that the expression ? has type `m≤m ≡ y`

% So we're not able to proceed without K, it would seem.

% At the same time, I noticed I'd have to prove other, more complex properties
% about `≤`, and it was all looking quite daunting, until I realized that I hadn't
% had to prove *any* of these things for the exponentiation case: why not? Well
% because the precise proofs we needed were given to us, in the result of the
% `compare` function. What we got in `≤-compare` were just proofs about the
% lengths, when really we should have looked for proofs about the inequalities
% themselves.

% # Indexed Ordering

% So what would a type for comparisons of the inequalities look like? Well, if we
% mimic the one for numbers:

% ```agda
% data Ordering : ℕ → ℕ → Set where
%   less    : ∀ m k → Ordering m (suc (m + k))
%   equal   : ∀ m   → Ordering m m
%   greater : ∀ m k → Ordering (suc (m + k)) mv
% ```

% We can see that we'll need some notion of `+` for an inequality: as it turns
% out, *transitivity* is the function we're looking for.

% ```agda
% infixl 6 _⋈_
% _⋈_ : ∀ {x y z} → x ≤ y → y ≤ z → x ≤ z
% xs ⋈ m≤m = xs
% xs ⋈ (≤-s ys) = ≤-s (xs ⋈ ys)
% ```

% With that, we can write the new `Ordering` type:

% ```agda
% data Ordering {n : ℕ} : ∀ {i j}
%                       → (i≤n : i ≤ n)
%                       → (j≤n : j ≤ n)
%                       → Set
%                       where
%   _<_ : ∀ {i j-1}
%       → (i≤j-1 : i ≤ j-1)
%       → (j≤n : suc j-1 ≤ n)
%       → Ordering (≤-s i≤j-1 ⋈ j≤n) j≤n
%   _>_ : ∀ {i-1 j}
%       → (i≤n : suc i-1 ≤ n)
%       → (j≤i-1 : j ≤ i-1)
%       → Ordering i≤n (≤-s j≤i-1 ⋈ i≤n)
%   eq : ∀ {i} → (i≤n : i ≤ n) → Ordering i≤n i≤n
% ```

% The compare function (written here as an operator):

% ```agda
% _∺_ : ∀ {i j n}
%     → (x : i ≤ n)
%     → (y : j ≤ n)
%     → Ordering x y
% m≤m ∺ m≤m = eq m≤m
% m≤m ∺ ≤-s y = m≤m > y
% ≤-s x ∺ m≤m = x < m≤m
% ≤-s x ∺ ≤-s y with x ∺ y
% ≤-s .(≤-s i≤j-1 ⋈ y) ∺ ≤-s y                | i≤j-1 < .y = i≤j-1 < ≤-s y
% ≤-s x                ∺ ≤-s .(≤-s j≤i-1 ⋈ x) | .x > j≤i-1 = ≤-s x > j≤i-1
% ≤-s x                ∺ ≤-s .x               | eq .x = eq (≤-s x)
% ```

% Again, closely mimicking the sparse exponent encoding, we'll write a "power" and
% normalizing constructor:

% ```agda
% _Π↑_ : ∀ {n m} → Poly n → (suc n ≤ m) → Poly m
% (xs Π i≤n) Π↑ n≤m = xs Π (≤-s i≤n ⋈ n≤m)

% infixr 4 _Π↓_
% _Π↓_ : ∀ {i n} → Coeffs i → suc i ≤ n → Poly n
% []                       Π↓ i≤n = Κ 0# Π z≤n
% (x ≠0 Δ zero  ∷ [])      Π↓ i≤n = x Π↑ i≤n
% (x₁   Δ zero  ∷ x₂ ∷ xs) Π↓ i≤n = Σ (x₁ Δ zero  ∷ x₂ ∷ xs) Π i≤n
% (x    Δ suc j ∷ xs)      Π↓ i≤n = Σ (x  Δ suc j ∷ xs) Π i≤n
% ```

% Finally, we can write our addition (I've omitted the parts which haven't
% changed):

% ```agda
% infixl 6 _⊞_
% _⊞_ : ∀ {n} → Poly n → Poly n → Poly n
% (xs Π i≤n) ⊞ (ys Π j≤n) = ⊞-match (i≤n ∺ j≤n) xs ys

% ⊞-match : ∀ {i j n}
%       → {i≤n : i ≤ n}
%       → {j≤n : j ≤ n}
%       → Ordering i≤n j≤n
%       → FlatPoly i
%       → FlatPoly j
%       → Poly n
% ⊞-match (eq i&j≤n)    (Κ x)  (Κ y)  = Κ (x + y)         Π  i&j≤n
% ⊞-match (eq i&j≤n)    (Σ xs) (Σ ys) = ⊞-coeffs    xs ys Π↓ i&j≤n
% ⊞-match (i≤j-1 < j≤n)  xs    (Σ ys) = ⊞-inj i≤j-1 xs ys Π↓ j≤n
% ⊞-match (i≤n > j≤i-1) (Σ xs)  ys    = ⊞-inj j≤i-1 ys xs Π↓ i≤n
% ```

% # Evaluation

% For evaluating our polynomials, we'll make use of "Horner's rule": this is just
% a simple algorithm that cuts down on the number of multiplications required. In
% the gappy encoding, it would have looked like this:

% ```agda
% ⟦_⟧ : Poly → Carrier → Carrier
% ⟦ xs ⟧ ρ = foldr (λ y ys → y + ys * ρ) 0# xs
% ```

% For the gapless, we get the following:

% ```agda
% ⟦_⟧ : Poly → Carrier → Carrier
% ⟦ xs ⟧ ρ = foldr coeff-eval 0# xs
%   where
%   coeff-eval : Coeff × ℕ → Carrier → Carrier
%   coeff-eval (0≠ x , i) xs = (x + xs * ρ) * ρ ^ i
% ```

% Finally, if we want to handle multiple variables, we'll get the following:

% ```agda
% drop : ∀ {i n} → i ≤ n → Vec Carrier n → Vec Carrier i
% drop m≤m Ρ = Ρ
% drop (≤-s si≤n) (_ ∷ Ρ) = drop si≤n Ρ

% vec-uncons : ∀ {n} → Vec Carrier (suc n) → Carrier × Vec Carrier n
% vec-uncons (x ∷ xs) = x , xs

% drop-1 : ∀ {i n} → suc i ≤ n → Vec Carrier n → Carrier × Vec Carrier i
% drop-1 si≤n xs = vec-uncons (drop si≤n xs)

% mutual
%   Σ⟦_⟧ : ∀ {n} → Coeffs n → (Carrier × Vec Carrier n) → Carrier
%   Σ⟦ x ≠0 Δ i ∷ xs ⟧ (ρ , Ρ) = (⟦ x ⟧ Ρ + Σ⟦ xs ⟧ (ρ , Ρ) * ρ) * ρ ^ i
%   Σ⟦ [] ⟧ _ = 0#

%   ⟦_⟧ : ∀ {n} → Poly n → Vec Carrier n → Carrier
%   ⟦ Κ x  Π i≤n ⟧ _ = x
%   ⟦ Σ xs Π i≤n ⟧ Ρ = Σ⟦ xs ⟧ (drop-1 i≤n Ρ)
% ```

% We see here again that the choice of inequality was the right one: we only pay
% for the amount of the vector that we drop, rather than for the size of the tail,
% as we might using another one of the inequalities.

% # Writing the Proofs

% We're going to prove homomorphism according to some
% [setoid](https://en.wikipedia.org/wiki/Setoid): an equivalence relation over
% some set. This relation could be propositional equality (for instance), but it's
% not necessary: it could instead be something more exotic which we'll see later.

% As an equivalence relation, it needs the following three functions:

% ```agda
% -- Reflexivity
% refl : ∀ {x} → x ≈ x
% -- Symmetry
% sym : ∀ {x y} → x ≈ y → y ≈ x
% -- Transitivity
% trans : ∀ {x y z} → x ≈ y → y ≈ z → x ≈ z
% ```

% And, as the type we're working with is a ring, it will need to obey the ring
% laws, according to the above relation. For example:

% ```agda
% *-comm : ∀ x y → x * y ≈ y * x
% ```

% There's one more thing: because this is a setoid, you don't get *congruence*
% automatically. This means that the user has to supply a further two functions:

% ```agda
% +-cong : ∀ {x₁ x₂ y₁ y₂} → x₁ ≈ x₂ → y₁ ≈ y₂ → x₁ + y₁ ≈ x₂ + y₂
% *-cong : ∀ {x₁ x₂ y₁ y₂} → x₁ ≈ x₂ → y₁ ≈ y₂ → x₁ * y₁ ≈ x₂ * y₂
% ```

% With all of this stuff, we could go ahead and prove everything we need at this
% point. But it's more than a little painful: for instance, we'll need later on a
% definition for exponentiation.

% ```agda
% infixr 8 _^_
% _^_ : Carrier → ℕ → Carrier
% x ^ zero = 1#
% x ^ suc n = x * x ^ n
% ```

% A property we'll rely on is the following familiar identity:

% $$ x^i x^j = x^{i + j} $$

% A proof of this fact looks like this:

% ```agda
% pow-add : ∀ x i j
%         → x ^ i * x ^ j ≈ x ^ (i ℕ.+ j)
% pow-add : ∀ x i j → x ^ i * x ^ j ≈ x ^ (i ℕ.+ j)
% pow-add x zero j = *-identityˡ (x ^ j)
% pow-add x (suc i) j =
%   *-assoc x (x ^ i) (x ^ j)
%     ⟨ trans ⟩
%   (refl ⟨ *-cong ⟩ pow-add x i j)
% ```

% The brackets (`⟨ ⟩`) work like backticks in Haskell: they turn a function with
% two arguments into an infix operator.

% The issue is that it's hard to see what's going on. First things first, the
% pattern "`refl ⟨ *-cong ⟩ prf`" will be used often: basically, it means "apply
% `prf` to the right-hand-side of the `*`". To cut down on noise, we can define
% some operators to do that for us:

% ```agda
% infixr 1 ≪+_ +≫_ ≪*_ *≫_
% ≪+_ : ∀ {x₁ x₂ y} → x₁ ≈ x₂ → x₁ + y ≈ x₂ + y
% ≪+ prf = +-cong prf refl

% +≫_ : ∀ {x y₁ y₂} → y₁ ≈ y₂ → x + y₁ ≈ x + y₂
% +≫_ = +-cong refl

% ≪*_ : ∀ {x₁ x₂ y} → x₁ ≈ x₂ → x₁ * y ≈ x₂ * y
% ≪* prf = *-cong prf refl

% *≫_ : ∀ {x y₁ y₂} → y₁ ≈ y₂ → x * y₁ ≈ x * y₂
% *≫_ = *-cong refl
% ```

% "`≪+`" means: "apply this proof to the left hand side of the plus". Because
% they're right-associative, they can be chained, so if we wanted to drill down
% into some large expression, we could write "`≪+ ≪+ *≫ ≪* prf`", which would mean
% "to the left of the plus, left again, right of the times, and left of the
% times", like directions.

% That doesn't solve the real issue, though: the proof is hard to read because we
% don't know what's going on inside it. If we were to write it out with a pen and
% paper, we'd write the state of the expression after each rewrite along with the
% rules we were using to justify it. Luckily, we can do the same in Agda, like so:

% ```agda
% pow-add : ∀ x i j → x ^ i * x ^ j ≈ x ^ (i ℕ.+ j)
% pow-add x zero j = *-identityˡ (x ^ j)
% pow-add x (suc i) j =
%   begin
%     x ^ suc i * x ^ j
%   ≡⟨⟩
%     (x * x ^ i) * x ^ j
%   ≈⟨ *-assoc x (x ^ i) (x ^ j) ⟩
%     x * (x ^ i * x ^ j)
%   ≈⟨ *≫ pow-add x i j ⟩
%     x * x ^ (i ℕ.+ j)
%   ≡⟨⟩
%     x ^ suc (i ℕ.+ j)
%   ∎
% ```

% The syntax works like this: `begin` signals the start of the proof, `∎` the end.
% In between, you write the state of the expression, interspersed with the rules
% that allow you too rewrite it from one form to the next. `≡⟨⟩` is the simplest
% rule: you use it when a rewrite is definitionally equal. So on our first
% rewrite, we're just using the definition of `^`. Then, `≈⟨ prf ⟩` applies the
% equality inside the brackets.

% This way of writing proofs is very powerful: it makes what might seem
% intractably complex manageable. 

% Also, while it may *seem* like this is a special language feature, these are all
% actually just normal operators, defined in the standard library.

% # List Homomorphism

% We use list algebras too make the proofs cleaner.

% @mu_algebra_2009

% # Setoid

% I mentioned that the notion of equality we were using was more general than
% propositional, and that we could use it more flexibly in different contexts.

% ## Tracing

% First off, we can look at lists (again). The three things required by an
% equivalence relation: reflexivity, transitivity, and symmetry, are all familiar
% functions on lists: empty lists, concatenation, and reversal. For propositional
% equality, of course, these lists are purely abstract. For us, though, we can
% fill them with whatever we want.

% [Wolfram
% Alpha](http://www.wolframalpha.com/examples/pro-features/step-by-step-solutions/v)
% will let you perform various manipulations on equations and expressions, even
% with abstract variables. What's most useful about the manipulations is that it
% will also provide "step-by-step solutions" of the transformations.

% We can do the same here, with a custom setoid:

% ```agda
% infix 4 _≡⋯≡_
% infixr 5 _≡⟨_⟩_
% data _≡⋯≡_ : A → A → Set a where
%   [refl] : ∀ {x} →  x ≡⋯≡ x
%   _≡⟨_⟩_ : ∀ {x} y {z} → String → y ≡⋯≡ z → x ≡⋯≡ z
%   cong₁ : ∀ {x y z} {f : A → A}
%         → String
%         → x ≡⋯≡ y
%         → f y ≡⋯≡ z
%         → f x ≡⋯≡ z
%   cong₂ : ∀ {x₁ x₂ y₁ y₂ z} {f : A → A → A}
%         → String
%         → x₁ ≡⋯≡ x₂
%         → y₁ ≡⋯≡ y₂
%         → f x₂ y₂ ≡⋯≡ z
%         → f x₁ y₁ ≡⋯≡ z
% ```

% ## Isomorphism

% The other loosish notion of equivalence is *isomorphism*
% [@coquand_isomorphism_2013].

% # Correct by construction

% The Coq style is too write the implementation is correctness by construction
% [@geuvers_automatically_2017].

% ```agda
% infixr 0 ⟦⟧⇐_ ⟦_∷_⟨_⟩⟧⇐_
% data Poly (expr : Carrier) : Set (a ⊔ ℓ) where
%   ⟦⟧⇐_  : expr ≋ 0# → Poly expr
%   ⟦_∷_⟨_⟩⟧⇐_ : ∀ x xs → Poly xs → expr ≋ (λ ρ → x Coeff.+ ρ Coeff.* xs ρ) → Poly expr

% _⊞_ : ∀ {x y} → Poly x → Poly y → Poly (x + y)
% (⟦⟧⇐ xp) ⊞ (⟦⟧⇐ yp) = ⟦⟧⇐ xp ⟨ +-cong ⟩ yp ⟨ trans ⟩ +-identityˡ _
% (⟦⟧⇐ xp) ⊞ (⟦ y ∷ ys ⟨ ys′ ⟩⟧⇐ yp) = ⟦ y ∷ ys ⟨ ys′ ⟩⟧⇐ xp ⟨ +-cong ⟩ yp ⟨ trans ⟩ +-identityˡ _
% (⟦ x ∷ xs ⟨ xs′ ⟩⟧⇐ xp) ⊞ (⟦⟧⇐ yp) = ⟦ x ∷ xs ⟨ xs′ ⟩⟧⇐ xp ⟨ +-cong ⟩ yp ⟨ trans ⟩ +-identityʳ _
% (⟦ x ∷ xs ⟨ xs′ ⟩⟧⇐ xp) ⊞ (⟦ y ∷ ys ⟨ ys′ ⟩⟧⇐ yp) = ⟦ x Coeff.+ y ∷ xs + ys ⟨ xs′ ⊞ ys′ ⟩⟧⇐
%   xp ⟨ +-cong ⟩ yp ⟨ trans ⟩  λ ρ → +-distrib _ _ _ _ ρ
% ```

% # Other structures

% Semiring has a free equivalent: [@rivas_monoids_2015]
\bibliographystyle{IEEEtranS}
\bibliography{horners-rule.bib}
\end{document}