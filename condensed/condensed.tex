%% For double-blind review submission, w/o CCS and ACM Reference (max submission space)
\documentclass[acmsmall,review,anonymous]{acmart}\settopmatter{printfolios=true,printccs=false,printacmref=false}
%% For double-blind review submission, w/ CCS and ACM Reference
%\documentclass[acmsmall,review,anonymous]{acmart}\settopmatter{printfolios=true}
%% For single-blind review submission, w/o CCS and ACM Reference (max submission space)
%\documentclass[acmsmall,review]{acmart}\settopmatter{printfolios=true,printccs=false,printacmref=false}
%% For single-blind review submission, w/ CCS and ACM Reference
%\documentclass[acmsmall,review]{acmart}\settopmatter{printfolios=true}
%% For final camera-ready submission, w/ required CCS and ACM Reference
%\documentclass[acmsmall]{acmart}\settopmatter{}


%% Journal information
%% Supplied to authors by publisher for camera-ready submission;
%% use defaults for review submission.
\acmJournal{PACMPL}
\acmVolume{1}
\acmNumber{CONF} % CONF = POPL or ICFP or OOPSLA
\acmArticle{1}
\acmYear{2018}
\acmMonth{1}
\acmDOI{} % \acmDOI{10.1145/nnnnnnn.nnnnnnn}
\startPage{1}

%% Copyright information
%% Supplied to authors (based on authors' rights management selection;
%% see authors.acm.org) by publisher for camera-ready submission;
%% use 'none' for review submission.
\setcopyright{none}
%\setcopyright{acmcopyright}
%\setcopyright{acmlicensed}
%\setcopyright{rightsretained}
%\copyrightyear{2018}           %% If different from \acmYear

%% Bibliography style
\bibliographystyle{ACM-Reference-Format}
%% Citation style
%% Note: author/year citations are required for papers published as an
%% issue of PACMPL.
\citestyle{acmauthoryear}   %% For author/year citations


%%%%%%%%%%%%%%%%%%%%%%%%%%%%%%%%%%%%%%%%%%%%%%%%%%%%%%%%%%%%%%%%%%%%%%
%% Note: Authors migrating a paper from PACMPL format to traditional
%% SIGPLAN proceedings format must update the '\documentclass' and
%% topmatter commands above; see 'acmart-sigplanproc-template.tex'.
%%%%%%%%%%%%%%%%%%%%%%%%%%%%%%%%%%%%%%%%%%%%%%%%%%%%%%%%%%%%%%%%%%%%%%


%% Some recommended packages.
\usepackage{booktabs}   %% For formal tables:
                        %% http://ctan.org/pkg/booktabs
\usepackage{subcaption} %% For complex figures with subfigures/subcaptions
                        %% http://ctan.org/pkg/subcaption
\usepackage{catchfilebetweentags} %% For importing code snippets

%%%%%%%%%%%%%%%%%%%%%%%%%%%%%%%%%%%%%%%%%%%%%%%%%%%%%%%%%%%%%%%%%%%%%%%%%%%%%%%%
%% Agda special Characters
%%%%%%%%%%%%%%%%%%%%%%%%%%%%%%%%%%%%%%%%%%%%%%%%%%%%%%%%%%%%%%%%%%%%%%%%%%%%%%%%
\usepackage{amssymb}
\usepackage{turnstile}
\usepackage{bbm}
\usepackage[greek, english]{babel}
\usepackage{MnSymbol}
\usepackage{stmaryrd}
\usepackage{csquotes}
\newcommand\doubleplus{+\kern-1.3ex+\kern0.8ex}
\newcommand\mdoubleplus{\ensuremath{\mathbin{+\mkern-8mu+}}}
\makeatletter
\newcommand\incircbin
{%
  \mathpalette\@incircbin
}
\newcommand\@incircbin[2]
{%
  \mathbin%
  {%
    \ooalign{\hidewidth$#1#2$\hidewidth\crcr$#1\bigcirc$}%
  }%
}
\newcommand{\oeq}{\ensuremath{\incircbin{=}}}
\makeatother
\makeatletter
\newcommand\insquarebin
{%
  \mathpalette\@insquarebin
}
\newcommand\@insquarebin[2]
{%
  \mathbin%
  {%
    \ooalign{\hidewidth$#1#2$\hidewidth\crcr$#1\bigbox$}%
  }%
}
\newcommand{\sqtri}{\ensuremath{\insquarebin{\triangle}}}
\makeatother
\usepackage{ucs}
\DeclareUnicodeCharacter{8759}{\ensuremath{\squaredots}}
\DeclareUnicodeCharacter{951}{\textgreek{\texteta}}
\DeclareUnicodeCharacter{737}{\ensuremath{^\text{l}}}
\DeclareUnicodeCharacter{691}{\ensuremath{^\text{r}}}
\DeclareUnicodeCharacter{7523}{\ensuremath{_\text{r}}}
\DeclareUnicodeCharacter{8718}{\ensuremath{\blacksquare}}
\DeclareUnicodeCharacter{957}{\textgreek{\textnu}}
\DeclareUnicodeCharacter{961}{\textgreek{\textrho}}
\DeclareUnicodeCharacter{929}{\textgreek{\textRho}}
\DeclareUnicodeCharacter{954}{\textgreek{\textkappa}}
\DeclareUnicodeCharacter{10214}{\ensuremath{\lsem}}
\DeclareUnicodeCharacter{10215}{\ensuremath{\rsem}}
\DeclareUnicodeCharacter{8857}{\mdoubleplus}
\DeclareUnicodeCharacter{8860}{\oeq}
\DeclareUnicodeCharacter{9043}{\ensuremath{\sqtri}}
\DeclareUnicodeCharacter{928}{\textgreek{\textPi}}
\DeclareUnicodeCharacter{922}{\textgreek{\textKappa}}
\DeclareUnicodeCharacter{931}{\textgreek{\textSigma}}
\DeclareUnicodeCharacter{916}{\textgreek{\textDelta}}
\DeclareUnicodeCharacter{8779}{\ensuremath{\backtriplesim}}
\DeclareUnicodeCharacter{8799}{\ensuremath{\stackrel{?}{=}}}
\DeclareUnicodeCharacter{10181}{\ensuremath{\lbag}}
\DeclareUnicodeCharacter{10182}{\ensuremath{\rbag}}
\DeclareUnicodeCharacter{8760}{\ensuremath{-}}
\usepackage[references]{agda}
%%%%%%%%%%%%%%%%%%%%%%%%%%%%%%%%%%%%%%%%%%%%%%%%%%%%%%%%%%%%%%%%%%%%%%%%%%%%%%%%


\begin{document}

%% Title information
\title[Reading and Writing Arithmetic]{Reading and Writing Arithmetic: Automating Ring Equalities
  in Agda}

%% Author information
%% Contents and number of authors suppressed with 'anonymous'.
%% Each author should be introduced by \author, followed by
%% \authornote (optional), \orcid (optional), \affiliation, and
%% \email.
%% An author may have multiple affiliations and/or emails; repeat the
%% appropriate command.
%% Many elements are not rendered, but should be provided for metadata
%% extraction tools.

%% Author with single affiliation.
\author{Donnacha Oisín Kidney}
\authornote{with author1 note}          %% \authornote is optional;
                                        %% can be repeated if necessary
\orcid{nnnn-nnnn-nnnn-nnnn}             %% \orcid is optional
\affiliation{
  \position{Position1}
  \department{Department1}              %% \department is recommended
  \institution{Institution1}            %% \institution is required
  \streetaddress{Street1 Address1}
  \city{City1}
  \state{State1}
  \postcode{Post-Code1}
  \country{Country1}                    %% \country is recommended
}
\email{first1.last1@inst1.edu}          %% \email is recommended

%% Author with two affiliations and emails.
\author{First2 Last2}
\authornote{with author2 note}          %% \authornote is optional;
                                        %% can be repeated if necessary
\orcid{nnnn-nnnn-nnnn-nnnn}             %% \orcid is optional
\affiliation{
  \position{Position2a}
  \department{Department2a}             %% \department is recommended
  \institution{Institution2a}           %% \institution is required
  \streetaddress{Street2a Address2a}
  \city{City2a}
  \state{State2a}
  \postcode{Post-Code2a}
  \country{Country2a}                   %% \country is recommended
}
\email{first2.last2@inst2a.com}         %% \email is recommended
\affiliation{
  \position{Position2b}
  \department{Department2b}             %% \department is recommended
  \institution{Institution2b}           %% \institution is required
  \streetaddress{Street3b Address2b}
  \city{City2b}
  \state{State2b}
  \postcode{Post-Code2b}
  \country{Country2b}                   %% \country is recommended
}
\email{first2.last2@inst2b.org}         %% \email is recommended



%% Abstract
%% Note: \begin{abstract}...\end{abstract} environment must come
%% before \maketitle command
\begin{abstract}
  We present a new library which automates the construction of equivalence
  proofs between polynomials over commutative rings and semirings in the
  programming language Agda \cite{norell_dependently_2008}. It is asymptotically
  faster than Agda's existing solver. We use reflection to provide a simple
  interface to the solver, and demonstrate a novel use of the constructed
  relations: step-by-step solutions.
\end{abstract}


%% 2012 ACM Computing Classification System (CSS) concepts
%% Generate at 'http://dl.acm.org/ccs/ccs.cfm'.
\begin{CCSXML}
<ccs2012>
<concept>
<concept_id>10011007.10011006.10011008</concept_id>
<concept_desc>Software and its engineering~General programming languages</concept_desc>
<concept_significance>500</concept_significance>
</concept>
<concept>
<concept_id>10003456.10003457.10003521.10003525</concept_id>
<concept_desc>Social and professional topics~History of programming languages</concept_desc>
<concept_significance>300</concept_significance>
</concept>
</ccs2012>
\end{CCSXML}

\ccsdesc[500]{Software and its engineering~General programming languages}
\ccsdesc[300]{Social and professional topics~History of programming languages}
%% End of generated code


%% Keywords
%% comma separated list
\keywords{proof automation, equivalence, proof by reflection, step-by-step solutions}


%% \maketitle
%% Note: \maketitle command must come after title commands, author
%% commands, abstract environment, Computing Classification System
%% environment and commands, and keywords command.
\maketitle

\begin{figure}[h]
  \begin{subfigure}[b]{\linewidth}
    \centering
    \ExecuteMetaData[../Introduction.tex]{lemma}
    \label{ring-lemma}
  \end{subfigure}
  \begin{subfigure}[b]{.5\linewidth}
    \ExecuteMetaData[../Introduction.tex]{proof}
    \caption{A Tedious Proof}
    \label{ring-proof}
  \end{subfigure}%
  \begin{subfigure}[b]{.3\linewidth}
    \centering
    \ExecuteMetaData[../Introduction.tex]{solver}
    \caption{The Solver}
    \label{the-solver}
  \end{subfigure}
  \caption{Comparison Between A Manual Proof and The Automated Solver}
  \label{comparison}
\end{figure}

\begin{figure}
  \ExecuteMetaData[../Introduction.tex]{old-solver}
  \caption{The Old Solver}
  \label{old-solver}
\end{figure}
\section{Introduction}
Doing mathematics in a dependently-typed programming languages like Agda
\cite{norell_dependently_2008} has a reputation for being tedious, awkward, and
difficult. Even simple arithmetic identities, like the one in
Fig.~\ref{comparison}, require fussy proofs (Fig.~\ref{ring-proof}).

This need not be the case! With some carefully-designed tools, mathematics in
Agda can be easy, friendly, and fun. This work describes one such tool, to deal
with equalities over commutative rings and semirings.
\subsection{Contributions}
We present a library which automates the solving of equalities over commutative
rings and semirings in Agda. In writing this library, we had three main goals:
\begin{description}
  \item[Friendliness and Ease of Use] Proofs like the one in
    Fig.~\ref{ring-proof} aren't just boring: they're \emph{difficult}.
    The programmer needs to remember the particular syntax for each step (``is
    it \(\AgdaFunction{+-comm}\) or \(\AgdaFunction{+-commutative}\)?''), and
    often they have to put up with poor error messages.

    Even though Agda's standard library \cite{danielsson_agda_2018} currently
    has a ring solver, its interface is almost as verbose as the manual proof,
    and it requires users write the goal twice, once in the signature and again
    in the specific syntax used by the solver (Fig.~\ref{old-solver}).

    Our solver strives to be as easy to use as possible: the high-level
    interface is simple (Fig.~\ref{the-solver}), we don't require anything
    of the user other than an implementation of one of the supported algebras,
    and effort is made to generate useful error messages.
  \item[Efficiency] Typechecking dependently-typed code is a costly task.
    Automated solvers, like the one presented here, can seriously add to that
    cost, occasionally to the extent that some identities are simply infeasible
    to prove.

    Polynomials are represented internally in sparse Horner normal form.
    Manipulation of this internal representation uses many of the same
    optimizations as in \cite{gregoire_proving_2005}, although their
    implementation proved to be quite difficult in Agda. Furthermore, we found
    that the real performance bottleneck lied elsewhere, so overall our strategy
    for optimization was quite different. The end result is that our solver is
    asymptotically faster than Agda's current.
  \item[Educational]

    
    

\end{description}

%% Acknowledgments
\begin{acks}                            %% acks environment is optional
                                        %% contents suppressed with 'anonymous'
  %% Commands \grantsponsor{<sponsorID>}{<name>}{<url>} and
  %% \grantnum[<url>]{<sponsorID>}{<number>} should be used to
  %% acknowledge financial support and will be used by metadata
  %% extraction tools.
  This material is based upon work supported by the
  \grantsponsor{GS100000001}{National Science
    Foundation}{http://dx.doi.org/10.13039/100000001} under Grant
  No.~\grantnum{GS100000001}{nnnnnnn} and Grant
  No.~\grantnum{GS100000001}{mmmmmmm}.  Any opinions, findings, and
  conclusions or recommendations expressed in this material are those
  of the author and do not necessarily reflect the views of the
  National Science Foundation.
\end{acks}


%% Bibliography
\bibliography{../bibliography.bib}
%% Appendix
\appendix
\section{Appendix}

Text of appendix \ldots

\end{document}
