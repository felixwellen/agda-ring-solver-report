\documentclass[draft, twocolumn]{article}
\usepackage[unicode,draft=false,hidelinks]{hyperref}
\usepackage{cite}
\usepackage{catchfilebetweentags}
\usepackage{amssymb}
\usepackage{turnstile}
\usepackage{bbm}
\usepackage[greek, english]{babel}
\usepackage{MnSymbol}
\usepackage{stmaryrd}
\usepackage{csquotes}
\newcommand\doubleplus{+\kern-1.3ex+\kern0.8ex}
\newcommand\mdoubleplus{\ensuremath{\mathbin{+\mkern-8mu+}}}
\makeatletter
\newcommand\incircbin
{%
  \mathpalette\@incircbin
}
\newcommand\@incircbin[2]
{%
  \mathbin%
  {%
    \ooalign{\hidewidth$#1#2$\hidewidth\crcr$#1\bigcirc$}%
  }%
}
\newcommand{\oeq}{\ensuremath{\incircbin{=}}}
\makeatother
\makeatletter
\newcommand\insquarebin
{%
  \mathpalette\@insquarebin
}
\newcommand\@insquarebin[2]
{%
  \mathbin%
  {%
    \ooalign{\hidewidth$#1#2$\hidewidth\crcr$#1\bigbox$}%
  }%
}
\newcommand{\sqtri}{\ensuremath{\insquarebin{\triangle}}}
\makeatother
\usepackage{ucs}
\DeclareUnicodeCharacter{8759}{\ensuremath{\squaredots}}
\DeclareUnicodeCharacter{951}{\textgreek{\texteta}}
\DeclareUnicodeCharacter{737}{\ensuremath{^\text{l}}}
\DeclareUnicodeCharacter{691}{\ensuremath{^\text{r}}}
\DeclareUnicodeCharacter{7523}{\ensuremath{_\text{r}}}
\DeclareUnicodeCharacter{8718}{\ensuremath{\blacksquare}}
\DeclareUnicodeCharacter{957}{\textgreek{\textnu}}
\DeclareUnicodeCharacter{961}{\textgreek{\textrho}}
\DeclareUnicodeCharacter{929}{\textgreek{\textRho}}
\DeclareUnicodeCharacter{954}{\textgreek{\textkappa}}
\DeclareUnicodeCharacter{10214}{\ensuremath{\lsem}}
\DeclareUnicodeCharacter{10215}{\ensuremath{\rsem}}
\DeclareUnicodeCharacter{8857}{\mdoubleplus}
\DeclareUnicodeCharacter{8860}{\oeq}
\DeclareUnicodeCharacter{9043}{\ensuremath{\sqtri}}
\DeclareUnicodeCharacter{928}{\textgreek{\textPi}}
\DeclareUnicodeCharacter{922}{\textgreek{\textKappa}}
\DeclareUnicodeCharacter{931}{\textgreek{\textSigma}}
\DeclareUnicodeCharacter{916}{\textgreek{\textDelta}}
\DeclareUnicodeCharacter{8779}{\ensuremath{\backtriplesim}}
\DeclareUnicodeCharacter{8799}{\ensuremath{\stackrel{?}{=}}}
\DeclareUnicodeCharacter{10181}{\ensuremath{\lbag}}
\DeclareUnicodeCharacter{10182}{\ensuremath{\rbag}}
\DeclareUnicodeCharacter{8760}{\ensuremath{-}}
\usepackage[utf8x]{inputenc}
\usepackage[T1]{fontenc}
\usepackage{autofe}
\usepackage[references]{agda}
\usepackage{bbding}
\setlength{\marginparwidth}{2cm}
\usepackage[obeyDraft]{todonotes}
\usepackage{lineno}
\setlength\linenumbersep{-0.5cm}
\usepackage{amsthm}
\theoremstyle{definition}
\newtheorem{definition}{Definition}[section]
\theoremstyle{definition}
\newtheorem{principle}{Principle}[section]
\usepackage{subcaption}
\usepackage{graphicx}
\usepackage{tikz}
\usetikzlibrary{decorations.pathmorphing}
\usetikzlibrary{snakes}
\usetikzlibrary{arrows}
\usetikzlibrary{cd}
\usepackage{forest}
\usepackage{pgfplots}
\usepackage{float}
\author{Donnacha Oisín Kidney}
\title{Automatically And Efficiently Illustrating Polynomial Equalities in
  Agda---Extended Abstract}
\begin{document}
\maketitle
\begin{abstract}
  We present a new library which automates the construction of equivalence
  proofs between polynomials over commutative rings and semirings in the
  programming language Agda\cite{norell_dependently_2008}. It is asymptotically
  faster than Agda's existing solver. We use Agda's reflection machinery to
  provide a simple interface to the solver, and demonstrate a novel use of the
  constructed relations: step-by-step solutions.

  The library is available at
  \href{https://oisdk.github.io/agda-ring-solver/README.html}{oisdk.github.io/agda-ring-solver/README.html}.
\end{abstract}
\tableofcontents
\section{Introduction}
\subsection{Formalized and Mechanized Mathematics}
Mathematics is often thought of as something which builds upwards: adding new
work and advances on the foundations of old. However, since the early twentieth
century, some have turned their attention in the other direction, examining
those foundations and ensuring they are really as solid as they seem.
Worryingly, serious flaws were found, Russell's paradox being perhaps the most
famous.

These flaws prompted what's known as the ``foundational crisis of mathematics'':
an effort to replace the old foundations with new, sound ones. One sketch of
these new foundations was proposed in the form of ``Hilbert's program''. 

Hilbert's program
was one such effort. It was too ambitious, however: Gödel's incompleteness
theorems showed that several of the goals were in fact \emph{impossible}.

These days, 

\bibliographystyle{IEEEtranS}
\bibliography{../bibliography.bib}
\end{document}